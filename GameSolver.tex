\documentclass{article}
\usepackage[utf8]{inputenc}
\usepackage{amsfonts}
\usepackage{listings}
\usepackage{xcolor}
\usepackage{graphicx}
\usepackage{float}

\definecolor{codegreen}{rgb}{0,0.6,0}
\definecolor{codegray}{rgb}{0.5,0.5,0.5}
\definecolor{codepurple}{rgb}{0.58,0,0.82}
\definecolor{backcolour}{rgb}{0.95,0.95,0.92}

\lstdefinestyle{mystyle}{
    backgroundcolor=\color{backcolour},   
    commentstyle=\color{codegreen},
    keywordstyle=\color{magenta},
    numberstyle=\tiny\color{codegray},
    stringstyle=\color{codepurple},
    basicstyle=\ttfamily\footnotesize,
    breakatwhitespace=false,         
    breaklines=true,                 
    captionpos=b,                    
    keepspaces=true,                 
    numbers=left,                    
    numbersep=5pt,                  
    showspaces=false,                
    showstringspaces=false,
    showtabs=false,                  
    tabsize=2
}

\lstset{style=mystyle}

\title{Finite Impartial Combinatorial Game Solver}
\author{Konstantinos Kritharidis}
\date{06/07/2021}

\begin{document}

\maketitle

\section{Introduction}

This document will analyse a code I've written to solve (theoretically) every finite impartial combinatorial game. More specifically:
\begin{itemize}
  \item combinatorial game\cite{combinatorial_wiki}: A sequential game with perfect information
  \item impartial\cite{mit}: A game in which the set of allowable moves depends only on the position of the game and not on which of the two players is moving
  \item finite\cite{mit}: There is a finite set of positions available in the game and the game eventually ends
  \item solve: Find if player 1 or player 2 wins and outputs the optimal strategy to achieve the win.
\end{itemize}
The code is heavily based on the notes\cite{mit} from MIT on the Theory of Impartial Games which explains how to solve every such game using the Sprague-Grundy Theorem\cite{sg}.

\section{Code Analysis}

\subsection{General Idea}

A game consists of a graph $G = (X, F)$ where\cite{mit}:
\begin{itemize}
\item $X$ is the set of all possible game positions
\item $F$ is a function that gives for each $x \in X$ a subset of possible $x$’s to move to, called followers. If $F(x)$ is empty, the position $x$ is terminal.
\item The start position is $x_0 \in X$. So player 1 moves first from $x_0$.
\item Players alternate moves. At position $x$, the player chooses from $y \in F(x)$.
\item The player confronted with the empty set $F(x)$ loses.
\end{itemize}
We will only look at graphs that are progressively bounded\cite{mit}, meaning that from every start position $x_0$, every path has finite length. In other words, the graph is finite and has no cycles.

The user inputs the initial game state, the possible moves and the terminal states. We construct the graph $G(X, F)$ and calculate the Sprague-Grundy (SG) value of each possible state. If the SG value of the initial state is non-zero, Player 1 wins if they play optimally. Otherwise, Player 2 wins if they play optimally. The code outputs the winning player and proceeds to show a possible winning strategy by starting at the initial state and climbing down the graph.

Assuming player 1 wins if they play optimally, we travel to a state which has an SG value of zero. Then, regardless of which move Player 2 makes, they are destined to lose, if Player 1 plays optimally, so they make a random move. The process repeats until a terminal state is reached. There, Player 2 will have no possible move and will, thus, lose the game. 

\section{Notes\cite{mit}}

To analyze the game we will need to use the function $mex$, or minimum excluded value, defined as
$$mex(S) = min{\{}n \in \mathbb{N} : n \notin S{\}}$$
The Sprague-Grundy function of a graph G = (X, F) is a function $g$ defined on $X$ that takes only non-negative integer values and is computed as follows:
$$g(x) = mex{\{}g(y) : y \in F(x){\}}$$
In words, the Sprague-Grundy (SG) is the smallest non-negative value not found among the SG values of the followers of $x$.

Since this function is defined recursively, we’ll need some base cases. We set all terminal nodes $x$ to have $g(x) = 0$. Then any nodes that have only terminal nodes as followers have $g(x) = 1$. In this way we can work our way through the graph until all nodes are assigned an SG value.

We can add any $n$ games $G_i$ together as follows:\\
To sum the games $G_1 = (X_1, F_1), G_2 = (X_2, F_2), \dots, G_n = (X_n, F_n)$,\\
$G(X, F) = G_1 + G_2 + \dots + G_n$ where:
\begin{itemize}
\item $X = X_1 \times X_2 \times X_3 \times \dots \times X_n$, or the set of all $n$-tuples such that $x_i \in X_i \forall i$
\item The maximum number of moves is the sum of the maximum number
of moves of each component game.
\end{itemize}

According to the Sprague-Grundy Theorem, the SG function for a sum of games on a graph is just the Nim sum of the SG functions of its components.\\
If $g_i$ is the Sprague-Grundy function of $G_i, i = 1, \dots, n$, then $G = G_1 + \dots + G_n$ has Sprague-Grundy function $g(x_1, \dots, x_n) = g_1(x_1) \oplus \dots \oplus g_n(x_n)$.

The winning strategy on a game that can be represented in such a graph would be to move to a vertex with $g(x) = 0$.

\begin{thebibliography}{9}

\bibitem{combinatorial_wiki} 
Combinatorial game theory, Wikipedia
\\\texttt{https://en.wikipedia.org/wiki/Combinatorial{\_}game{\_}theory}

\bibitem{mit} 
Theory of Impartial Games, MIT
\\\texttt{http://web.mit.edu/sp.268/www/nim.pdf}

\bibitem{sg} 
Sprague–Grundy theorem, Wikipedia
\\\texttt{https://en.wikipedia.org/wiki/Sprague-Grundy{\_}theorem}

\end{thebibliography}

\end{document}
